\chapter{Conlusioni}
\label{conclusioni}

Con questa tesi ci si è posto l'obbiettivo di apportare al codice dell'autenticatore Solo e alla libreria FIDO sviluppata da Yubico le opportune modifiche per garantire la \emph{survivability}. Il lavoro segue la direzione generale tracciata dal lavoro fatto dall'Ing. Magnanini e dal Prof. Ferretti, e dal lavoro già svolto in un'altra tesi precedente sulla sola libreria FIDO. Proprio in quest'ultima tesi vengono descritti risultati del tutto simili a quelli ottenuti in questa sede, segno che l'aggiunta del livello di sicurezza non costituisce elemento di riduzione delle performance. 

La soluzione è stata provata solamente in configurazione locale, motivo per il quale sarebbero necessarie ulteriori analisi. Inoltre servirebbe uno studio di fattibilità più approfondito per verificare quanto descritto nella sezione \ref{fattibilità} e l'effettiva implementazione in autenticatori hardware, con i costi che ne conseguono. L'aggiunta di un contatore limita le possibilità dell'hardware di fatto riducendo il rapporto costo microcontrollore/numero di credenziali. Si può stimare un costo a contatore ($21$ credenziali * $ 3$ livelli di sicurezza), con un prezzo dell'STM32L432 per grossi stock di $\$4.59673$, di circa $0,07$ cents, sicuramente non trascurabile per grandi numeri.

Ciò nonostante il la soluzione proposta testata secondo le modalità descritte risulta avere prestazioni adatte a scenari reali di autenticazione distribuita. 
\chapter{Conlusioni}
\label{conclusioni}

Con questa tesi ci si è posto l'obbiettivo di apportare al codice dell'autenticatore Solo e alla libreria FIDO sviluppata da Yubico le opportune modifiche per garantire la \emph{survivability}. La tesi segue la direzione di lavori preesistenti in ambito di autenticazione survivable \cite{magnanini:survivable} e del lavoro già svolto in un'altra tesi precedente sulla libreria FIDO. Proprio in quest'ultima tesi si descrivono risultati conformi a quanto ottenuto in questa sede, ovvero prestazioni adatte a scenari reali di autenticazione distribuita. Ciò è segno che l'aggiunta del livello di sicurezza non costituisce elemento di riduzione delle performance.

La soluzione è stata provata solamente in configurazione locale, motivo per il quale sarebbero necessarie ulteriori analisi prevedendo un autenticatore hardware reale e l'utilizzo di un Browser tramite cui simulare l'interazione dell'utente con un servizio Web. Inoltre, sarebbe interessante studiare la fattibilità per altre tipologie di microcontrollori e il rapporto costi/benefici che deriverebbe dall'adozione rispetto a quello preso in esame. 
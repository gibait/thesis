\chapter{Conoscenze di Base}
\label{conoscenze}

In questa sezione verranno introdotti i concetti necessari per proseguire con la lettura dei capitoli successivi. 

\section{Single Sign-On}
\label{sso}

% AGGIUNGERE ATTORI

Lo schema \emph{Single Sign-On} è un protocollo molto diffuso utilizzato per fare il login a servizi Web. L'utente non effettua l'autenticazione direttamente presso il servizio desiderato ma passa prima da un ente terzo. Questo ente è chiamato Identity Provider, fornisce un Identity Server presso cui l'utente deve autenticarsi per farsi rilasciare un \textbf{token} di autenticazione. Tale token verrà poi presentato al servizio desiderato così che possa finalmente essere fruito. 

\section{Survivability}
\label{surviv}

La centralizzazione dello schema SSO lascia spazio ad attacchi in cui un malintenzionato prenda il controllo dell'Identity Server e forgi token di autenticazione così da poter poi impersonare qualunque utente egli voglia. Vengono in aiuto gli schemi cosiddetti \emph{survivable SSO} che possono limitare tali problematiche sfruttando più Identity Server. Un singolo Identity Provider gestirà quindi più Identity Server e l'utente dovrà autenticarsi presso un sottoinsieme di questi, i quali rilasceranno poi una \textbf{attestazione} firmata collettivamente. 

La componente survivable risiede nel fatto che viene tollerato un certo numero di Identity Server violati e, di conseguenza, viene richiesta una attestazione in funzione di questo numero. Con una soglia di tolleranza di server maligni sufficiente si riesce a garantire l'integrità del meccanismo di autenticazione e un overhead, dovuto alla reiterazione dei passaggi, trascurabile.

\section{Passwordless}
\label{passwordless}

L'autenticazione passwordless è un metodo di autenticazione che permette ad un utente di effettuare il login ad un servizio senza la necessità di conoscere una password o più genericamente una conoscenza considerata segreta. Tipicamente utilizza una coppia di chiavi crittografiche, una \textbf{privata} e una \textbf{pubblica}: la prima viene generata e immagazzinata sul dispositivo dell'utente mentre la seconda viene inviata al server così che esso possa verificare l'autenticità dei messaggi ricevuti. La chiave privata, o segreta, non lascia mai il dispositivo su cui è stata creata e per accedervi è necessaria una qualche sorta di autenticazione da parte dell'utente \cite{wiki:passwordless}. L'autenticazione può riguardare:

\begin{itemize}
	\item Qualcosa che l'utente \textbf{possiede} come un telefono cellulare, un token OTP, un autenticatore hardware
	\item Qualcosa che l'utente \textbf{è} come l'impronta digitale, la scansione retinica o il riconoscimento vocale 
\end{itemize}

La registrazione passwordless e, conseguentemente, l'autenticazione vengono svolte seguendo un meccanismo \emph{challenge-response}: al pervenire di una richiesta di registrazione il server invia una cosidetta \emph{challenge}. L'utente che ha iniziato l'operazione avrà il compito di apporre, tramite propria chiave privata, una firma crittografica sulla challenge e di fornire in risposta al server la challenge firmata accompagnata dalla chiave pubblica. Così facendo il server verificherà l'autenticità della firma tramite la chiave appena ricevuta e in caso di esito positivo, la immagazzinerà. La fase di autenticazione verrà svolta in modo analogo con la differenza che il server è già in possesso della chiave pubblica e non sarà quindi necessario inviarla.

Come si può vedere non viene scambiato alcun segreto e l'unica interazione richiesta all'utente è quella in fase di firma della challenge. Anche allora l'utilizzatore non dovrà inserire codici o password ma semplicemente autenticarsi al dispositivo contenente la chiave privata tramite uno dei fattori sopra elencati. Sfruttando l'autenticazione passwordless è possibile sopperire alle criticità tipiche dei segreti a bassa entropia come le password, quali phishing, brute forcing etc. 

\section{FIDO}
\label{fido}

FIDO Alliance è un'associazione nata nel 2013 con lo scopo di migliorare i sistemi di autenticazione. Sono gli autori di \emph{FIDO}, un set di specifiche che include gli standard \textbf{CTAP} e \textbf{WebAuthn}. Nel corso degli anni vi sono stati un susseguirsi di iterazioni degli standard, prima conosciuta come Universal Authentication Factor per poi diventare Universal 2nd Factor e giungere infine alla versione corrente FIDO 2.0. 

Il protocollo CTAP definisce le specifiche tramite cui vengono programmati gli autenticatori crittografici per fare in modo che possano operare con un client. Il protocollo WebAuthn si occupa invece di definire le specifiche tramite cui standardizzare l'autenticazione a servizi web sfruttando le chiavi crittografiche. 

In particolare definiscono tutto il necessario per programmare un autenticatore e un server come: strutture dati, metodi, requisiti di funzionamento, encoding dei dati etc.

\subsection{Rilevamento tentativi di clonazione}
\label{fido:clonazione}

Compito dell'Identity Server è anche quello di rilevare eventuali tentativi di duplicazione dell'autenticatore fisico. Per fare ciò lo standard FIDO prevede un \textbf{contatore} gestito dall'autenticatore, sia esso globale o multiplo, che viene aggiornato ad ogni operazione avvenuta con successo. Il contatore prende il nome di \emph{signature counter}. Il server mantiene in memoria l'ultimo valore ricevuto e, all'interazione successiva, controlla che non vi siano discrepanze. In particolare se il valore ricevuto è minore di quello salvato in memoria dal server allora si può essere in presenza di un tentativo di clonazione. (L'accadere dell'inverso non è detto che costituisca un problema, motivo per il quale può essere utilizzato anche un contatore globale.) Ciò significa che un attaccante maligno ha duplicato l'autenticatore ad uno stato precedente mentre l'utilizzatore ha continuato ad autenticarsi incrementando il contatore. 
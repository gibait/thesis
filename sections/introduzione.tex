\chapter{Introduzione}
\label{intro}

Il Single Sign-On è un protocollo utilizzato per migliorare e semplificare il processo di autenticazione. Alleggerisce i servizi Web che lo utilizzano dall'onere di gestire la moltitudine di credenziali degli utenti, con i meccanismi di protezione necessari che ne conseguono. I protocolli di SSO possono autenticare gli utenti mediante protocolli passwordless. Grazie a questi protocolli un utente effettua l'autenticazione tramite un token hardware in grado di eseguire operazioni di autenticazione basate su crittografia asimmetrica. Questo permette di snellire e irrobustire il processo complessivo di autenticazione. Il limite principale risiede nella centralizzazione del protocollo SSO che non è in grado di tollerare intrusioni nell'infrastruttura che lo esegue.
Infatti, come mostrano attacchi recenti \cite{volexity:solarwinds} \cite{nsa:authentication}, gli attaccanti che riescono a compromettere l'infrastruttura di SSO sono in grado di impersonare utenti arbitrari.
Per mitigare questa problematica sono nati gli studi sulla cosiddetta SSO \emph{survivability}. Tale filone si occupa di irrobustire il protocollo SSO utilizzando un approccio distribuito e replicato. Si prevede un numero di attori malevoli di cui è possibile tollerare l'intromissione, numero da cui dipende il livello di sicurezza.

L'esigenza di questa tesi nasce dal fatto che i protocolli esistenti in ambito SSO \emph{survivable} non permettono di modificare il livello di sicurezza una volta che è stato stabilito durante la fase di creazione dell'infrastruttura. Questo impedisce ai vari servizi Web di richiedere alla stessa infrastruttura SSO livelli di sicurezza adeguati alle caratteristiche del servizio offerto. Si è affrontato dunque il problema dell'integrazione del livello di sicurezza all'interno dei protocolli Passwordless e SSO preesistenti. A tal fine, si è reso necessario implementare modifiche al codice di un autenticatore open source, cui hanno fatto seguito verifiche delle performance di autenticazione e un breve studio di fattibilità per l'implementazione su hardware reale.

La tesi è strutturata nel seguente modo: nel capitolo \ref{conoscenze} vengono introdotte le conoscenze necessarie per proseguire nella lettura dei successivi; il capitolo \ref{modellazione} di modellazione in cui vengono descritti gli attori che partecipano al protocollo di SSO proposto, il funzionamento dei protocolli citati e la soluzione studiata per integrare il livello di sicurezza; il capitolo \ref{dettagli} in cui vengono approfondite le modifiche apportate al codice degli attori presenti per realizzare l'integrazione; il capitolo \ref{prestazioni} di analisi dei dati raccolti, valutazione delle tempistiche e studio finale di fattibilità; il capitolo \ref{conclusioni} conclusivo con le considerazioni finali ed eventuali sviluppi futuri.
